\documentclass[a4paper, 12pt]{article}

\usepackage[utf8]{inputenc}
\usepackage[russian]{babel}
\usepackage{graphicx}
\graphicspath{{pictures/}}
\DeclareGraphicsExtensions{.pdf,.png,.jpg}
\usepackage[unicode, pdftex]{hyperref}
\usepackage{color}
\usepackage{float}
\setlength{\parindent}{5ex}
\setlength{\parskip}{1em}
\usepackage{indentfirst}

\begin{document}

\begin{titlepage}
  \thispagestyle{empty}
  \centerline {Санкт-Петербургский политехнический университет}
  \centerline { им. Петра Великого}
  \centerline { }
  \centerline {Институт прикладной математики и механики} 
  \centerline {Кафедра "Прикладная математика"}
  \vfill
  \centerline{\textbf{Отчёт}}
  \centerline{\textbf{по лабораторной работе №7}}
  \centerline{\textbf{по дисциплине}}
  \centerline{\textbf{"Математическая статистика"}}
  \vfill
  \hfill
  \begin{minipage}{0.45\textwidth}
  Выполнил студент:\\
  Дроздова Дарья Александровна\\
  группа: 3630102/80401 \\
  \\
  Проверил:\\
  к.ф.-м.н., доцент \\
  Баженов Александр Николаевич
  \end{minipage}
  \vfill
  \centerline {Санкт-Петербург}   
  \centerline {2021 г.}  
\end{titlepage}

\newpage
\setcounter{page}{2}
\tableofcontents

\newpage
\listoftables

\newpage
\section{Постановка задачи}

Сгенерировать выборку мощностью 100 элементов для нормального распределения $N(x,0,1)$. По данной выборке оценить параметры $\mu$ и $\sigma$ нормального закона методом максимального правдоподобия.

В качестве основной гипотезы $H_0$ будем считать, что сгенерированное распределение имеет вид $N(x, \widehat{\mu}, \widehat{\sigma})$. Проверить основную гипотезу, используя критерий согласия $\chi^2$. В качестве уровня значимости взять $\alpha=0.05$. Привести таблицу вычислений $\chi^2$.

Исследовать точность (чувствительность) критерия $\chi^2$ - сгенерировать выборки равномерного распределения и распределения Лапласа малого объема (20 элементов). Проверить их на нормальность

\newpage
\section{Теория}
\subsection{Метод максимального правдоподобия}

Пусть $x_1,...,x_n$ - случайная выборка из генеральной совокупности с плотностью вероятности $f(x, \theta)$. $L(x_1,...,x_n,\theta)$ - функция правдоподобия, представляющая собой совместную плотность вероятности независимых случайных величин $x_1,...,x_n$ и рассматриваемая как функция неизвестного параметра $\theta$:
$$
L(x_1,...,x_n,\theta)=f(x_1,\theta)f(x_2,\theta)...f(x_3,\theta)
$$

Оценкой максимального правдоподобия называется такое значение $\widehat{\theta}_{\mbox{пр}}$ из множества допустимых значений параметра $\theta$ для которого функция правдоподобия принимает наибольшее значение при заданных $x_1,...,x_n$:
$$
\widehat{\theta}_{\mbox{пр}} = \arg \max_\theta L(x_1,...,x_n,\theta)
$$

Если функция правдоподобия дважды дифференцируема то ее стационарные значения даются корнями уравнения 
$$
\frac{\partial L(x_1,...,x_n,\theta)}{\partial \theta} = 0
$$

Чаще проще искать максимум логарифма функции правдоподобия, т. е. решать следующее уравнение
$$
\frac{\partial \ln L}{\partial \theta} = 0
$$
которое называют уравнением правдоподобия.

\subsubsection{Оценка методом максимального правдоподобия математического ожидания $\mu$ и дисперсии $\sigma^2$ нормального распределения $N(\mu,\sigma)$}

Составим функцию правдоподобия
$$
L(x_1,...,x_n,\mu,\sigma) = \prod^n_{i=1} \frac{1}{\sigma\sqrt{2\pi}} \exp \left \{ - \frac{(x_i - \mu)^2}{2\sigma^2} \right \} = (2\pi\sigma^2)^{\frac{n}{2}} \exp \left \{ - \frac{1}{2\sigma^2} \sum^n_{i=1} (x_i - \mu)^2 \right \} 
$$
Тогда
$$
\ln L = -\frac{n}{2} \ln 2\pi - \frac{n}{2} \ln \sigma^2 - \frac{1}{2\sigma^2} \sum^n_{i=1} (x_i - \mu)^2
$$
Получим следующие уравнения правдоподобия
$$
\left \{ \begin{array}{ll}
\frac{\partial \ln L}{\partial m} = \frac{1}{\sigma^2} \sum^n_{i=1} (x_i - \widehat{\mu}) = \frac{n}{\sigma^2}(\overline{x} - \widehat{\mu}) = 0 \\
\frac{\partial \ln L}{\partial (\sigma^2)} = - \frac{n}{2\sigma^2} + \frac{1}{2(\sigma^2)^2}\sum^n_{i=1} (x_i - \widehat{\mu})^2 = \frac{n}{2(\sigma^2)^2} \left[ \frac{1}{n} \sum^n_{i=1} (x_i - \widehat{\mu})^2 - \widehat{\sigma}^2 \right] = 0
\end{array} \right.
$$
Откуда следует, что оценка методом максимального правдоподобия математического ожидания есть выборочное среднее, т. е.
\begin{equation}
\widehat{\mu}_{\mbox{мп}} = \overline{x}
\label{eq:1}
\end{equation}
а оценкой генеральной дисперсии - выборочная дисперсия
\begin{equation}
\widehat{\sigma}^2_{\mbox{мп}} = s^2
\label{eq:2}
\end{equation}
где 
$$
s^2 = \frac{1}{n}\sum^n_{i=1} (x_i - \overline{x})^2
$$

\subsection{Проверка гипотезы о законе распределения генеральной совокупности. Метод $\chi^2$}

Исчерпывающей характеристикой изучаемой случайной величины является её закон распределения. Поэтому естественно стремление построить этот закон приближённо на основе статистических данных.Сначала выдвигается гипотеза о виде закона распределения. После того как выбран вид закона, возникает задача оценивания его параметров и проверки (тестирования) закона в целом.Для проверки гипотезы о законе распределения применяются критерии согласия. Таких критериев существует много. Мы рассмотрим наиболее обоснованный и наиболее часто используемый в практике — критерий $\chi^2$, введённый К.Пирсоном (1900 г.) для случая, когда параметры
распределения известны.

Пусть выдвинута гипотеза $H_0$ о генеральном законе распределения с функцией распределения $F(x)$, которая не содержит неизвестных параметров. 

Разобьем генеральную совокупность на $k$ непересекающихся подмножеств, такие что
\begin{equation}
\begin{array}{c}
\bigtriangleup_i = (a_{i-1}, a_i] - \mbox{полуоткрытые промежутки}, i=\overline{2, k-1} \\
\bigtriangleup_1 = (-\infty, a_1] \\
\bigtriangleup_k = (a_{k-1}, + \infty)
\end{array}
\label{eq:3}
\end{equation}

В качестве числа $k$ берется число, вычисленное по следующей формуле
\begin{equation}
k \approx 1.72 \sqrt[3]{n}
\label{eq:4}
\end{equation}

Пусть $p_i = P(X \in \bigtriangleup_i), \; i = \overline{1,k}$, тогда
\begin{equation}
p_i = F(a_i) - F(a_{i-1}), \; a_0 = - \infty, a_k = + \infty, i = \overline{1,k}
\label{eq:5}
\end{equation}
заметим, что $\sum^k_{i=1} p_i = 1$ и будем считать, что все $p_i>0, \; i=\overline{1,k}$

Пусть $n_1,...,n_k$ - частоты попадания выборочных элементов в подмножества $\bigtriangleup_1,...,\bigtriangleup_k$ соответственно.

В случае справедливости гипотезы $H_0$ относительные частоты $\frac{n_i}{n}$ при большом $n$ должны быть близки к вероятности $p_i$, поэтому за меру отклонения выборочного распределения от гипотетического с функцией $F(x)$ естественно выбрать величину
$$
Z = \sum^k_{i=1}c_i \left( \frac{n_i}{n} - p_i \right)^2
$$
$c_i$ - положительные числа, веса. К. Пирсоном в качестве весов были выбраны числа 
\begin{equation}
c_i = \frac{n}{p_i}
\label{eq:6}
\end{equation}

Тогда получается статистика критерия $\chi^2$  К. Пирсона
$$
\chi^2 = \sum^k_{i=1}  \frac{(n_i - np_i)^2}{np_i}
$$

\subsubsection{Правило проверки гипотезы о законе распределения по методу $\chi^2$}

\begin{enumerate}
	\item Выбираем уровень значимости $\alpha$
	\item По таблице Приложение 1 находим квантиль $\chi^2_{1 - \alpha}(k-1)$ распределения хи-квадрат с $k-1$ степенями свободы порядка $1 - \alpha$
	\item Вычисляем вероятности $p_i$ по формуле (\ref{eq:5})
	\item Находим частоты $n_i, \; i=\overline{1,k}$
	\item Находим выборочное значение статистики критерия $\chi^2$:
	$$
	\chi^2_B = \sum^k_{i=1} \frac{(n_i - np_i)^2}{np_i}
	$$
	\item Сравниваем $\chi^2_B$ и квантиль $\chi^2_{1 - \alpha}(k-1)$:
	\begin{itemize}
		\item Если $\chi^2_B < \chi^2_{1 - \alpha}(k-1)$, то гипотеза $H_0$ на данном этапе проверки принимается
		\item Если $\chi^2_B \ge \chi^2_{1 - \alpha}(k-1)$, то гипотеза $H_0$ отвергается, выбирается одно из альтернативных распределений, и процедура проверки повторяется
	\end{itemize}
\end{enumerate}

\newpage
\section{Реализация}

Лабораторная работа выполнена на языке программирования Python в среде разработки PyCharm. Для построения распределений используется библиотека numpy и scipy.

Код программы расположен в репозитории GitHub по ссылке: \url{https://github.com/Drozdova-Daria/Math_Stat_Lab7}

\newpage
\section{Результаты}

\subsection{Метод максимального правдоподобия}

Оценка математического ожидания методом максимального правдоподобия (\ref{eq:1}):
$$
\widehat{\mu} \approx 0.0432
$$

Оценка генеральной дисперсии методом максимального правдоподобия (\ref{eq:2}):
$$
\widehat{\sigma} \approx 0.6797
$$

\subsection{Проверка гипотезы о законе распределения генеральной совокупности. Метод $\chi^2$}

\subsubsection{Нормальное распределение $n=100$}
Количество промежутков $k$ (\ref{eq:4}): $k = 8$

Уровень значимости $\alpha = 0.05$

Тогда квантиль по таблице в Приложении 1: $\chi^2_{1-\alpha}(k-1) = \chi^2_0.95(7) = 14.07$

\begin{table} [H]
\begin{center}
\begin{tabular}{|c|c|c|c|c|c|c|}
\hline 
$i$ & $\bigtriangleup_i$ (\ref{eq:3}) & $n_i$ & $p_i$ (\ref{eq:5}) & $np_i$ & $n_i - np_i$ & $\frac{(n_i-np_i)^2}{np_i}$ \\ 
\hline 
1 & $(-\infty, -1.01]$ & 22 & 0.1562 & 15.62 & 6.38 & 2.60 \\ 
\hline 
2 & $(-1.01, -0.58]$ & 13 & 0.1241 & 12.41 & 0.59 & 0.03 \\ 
\hline 
3 & $(-0.58, -0.15]$ & 13 & 0.1587 & 15.87 & -2.87 & 0.52 \\ 
\hline 
4 & $(-0.15, 0.28]$ & 11 & 0.1693 & 16.93 & -5.93 & 2.08 \\ 
\hline 
5 & $(0.28, 0.70]$ & 19 & 0.1507 & 15.07 & 3.93 & 1.02 \\ 
\hline 
6 & $(0.70, 1.13]$ & 9 & 0.1120 & 11.20 & -2.20 & 0.43 \\ 
\hline 
7 & $(1.13, 1.56]$ & 6 & 0.0695 & 6.95 & -0.95 & 0.13 \\ 
\hline 
8 & $(1.56, +\infty)$ & 7 & 0.0594 & 5.94 & 1.06 & 0.19 \\ 
\hline 
$\sum$ & - & 100 & 1.0000 & 100.00 & 0.00 & \color{red}{$6.99= \chi^2_B$} \\ 
\hline 
\end{tabular}
\caption{Вычисление $\chi^2_B$ при проверке гипотезы $H_0$ о нормальном законе распределения $N(x,\widehat{\mu}, \widehat{\sigma})$} 
\end{center}
\end{table}
Сравнивая $\chi^2_B = 6.99$ и $\chi^2_0.95(7) = 14.07$, получаем, что $\chi^2_B < \chi^2_0.95(4)$. Следовательно гипотезу $H_0$ на данном этапе можно принять.

\subsubsection{Равномерное распределение $n=20$}

Количество промежутков $k$ (\ref{eq:4}): $k = 5$

Уровень значимости $\alpha = 0.05$

Тогда квантиль по таблице в Приложение 1: $\chi^2_{1-\alpha}(k-1) = \chi^2_0.95(4) = 9.49$

\begin{table} [H]
\begin{center}
\begin{tabular}{|c|c|c|c|c|c|c|}
\hline 
$i$ & $\bigtriangleup_i$ (\ref{eq:3}) & $n_i$ & $p_i$ (\ref{eq:5}) & $np_i$ & $n_i - np_i$ & $\frac{(n_i-np_i)^2}{np_i}$ \\ 
\hline 
1 & $(-\infty, -1.5]$ & 0 & 0.0668 & 1.34 & -1.34 & 1.34 \\ 
\hline 
2 & $(-1.5, -0.48]$ & 8 & 0.2488 & 4.98 & 3.02 & 1.84 \\ 
\hline 
3 & $(-0.48, 0.54]$ & 7 & 0.3898 & 7.79 & -0.79 & 0.08 \\ 
\hline 
4 & $(0.54, 1.56]$ & 5 & 0.2352 & 4.70 & 0.29 & 0.02 \\ 
\hline 
5 & $(1.56, +\infty)$ & 0 & 0.0594 & 1.18 & -1.18 & 1.19 \\ 
\hline 
$\sum$ & - & 20 & 1.0000 & 20.00 & 0.00 & \color{red}{$4.46= \chi^2_B$} \\ 
\hline 
\end{tabular}
\caption{Вычисление $\chi^2_B$ при проверке гипотезы $H_0$ о равномерном законе распределения $U \left(x,\widehat{\mu}, 2\sqrt{3}\widehat{\sigma}\right)$} 
\end{center}
\end{table}
Сравнивая $\chi^2_B = 4.46$ и $\chi^2_0.95(4) = 9.49$, получаем, что $\chi^2_B < \chi^2_0.95(4)$. Следовательно гипотезу $H_0$ на данном этапе можно принять.

\subsubsection{Распределение Лапласа $n=20$}

Количество промежутков $k$ (\ref{eq:4}): $k = 5$

Уровень значимости $\alpha = 0.05$

Тогда квантиль по таблице в Приложение 1: $\chi^2_{1-\alpha}(k-1) = \chi^2_0.95(4) = 9.49$

\begin{table} [H]
\begin{center}
\begin{tabular}{|c|c|c|c|c|c|c|}
\hline 
$i$ & $\bigtriangleup_i$ (\ref{eq:3}) & $n_i$ & $p_i$ (\ref{eq:5}) & $np_i$ & $n_i - np_i$ & $\frac{(n_i-np_i)^2}{np_i}$ \\ 
\hline 
1 & $(-\infty, -1.5]$ & 1 & 0.0668 & 1.34 & -0.34 & 0.08 \\ 
\hline 
2 & $(-1.5, -0.48]$ & 6 & 0.2488 & 4.98 & 1.02 & 0.21 \\ 
\hline 
3 & $(-0.48, 0.54]$ & 4 & 0.3898 & 7.79 & -3.79 & 1.85 \\ 
\hline 
4 & $(0.54, 1.56]$ & 7 & 0.2352 & 4.70 & 2.29 & 1.12 \\ 
\hline 
5 & $(1.56, +\infty)$ & 2 & 0.0594 & 1.18 & 0.81 & 0.56 \\ 
\hline 
$\sum$ & - & 20 & 1.0000 & 20.00 & 0.00 & \color{red}{$3.82= \chi^2_B$} \\ 
\hline 
\end{tabular}
\caption{Вычисление $\chi^2_B$ при проверке гипотезы $H_0$ о законе распределения Лапласа $L \left(x,\widehat{\mu}, \frac{\widehat{\sigma}}{\sqrt{2}} \right)$} 
\end{center}
\end{table}
Сравнивая $\chi^2_B = 3.82$ и $\chi^2_0.95(4) = 9.49$, получаем, что $\chi^2_B < \chi^2_0.95(4)$. Следовательно гипотезу $H_0$ на данном этапе можно принять.

\newpage
\section{Обсуждение}

В результате проведенных исследований установили, что гипотезу $H_0$ можно принять для все исследуемых распределений.

Т. е. и при малых мощностях выборки критерий $\chi^2$ не почувствовал разницы между нормально распределенной величиной и распределенной по Лапласу или нормально распределенной величиной и равномерно распределенной величиной. Результат ожидаем, т. к. выборка довольно мала, законы схожи по форме и параметры масштаба и сдвига выбраны тоже так, чтобы законы максимально друг к другу приблизились.

\newpage
\section{Приложение 1}
\begin{figure}[h]
\center{\includegraphics[scale=0.7]{table}}
\end{figure}

\end{document}
